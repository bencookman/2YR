\section{Comparison with D-TDIBC}
% Maybe this should be in discussion chapter??

The work of \cite{douasbin2018DelayedtimeDomainImpedance} provides a similar one-dimensional model for outflow acoustics based on their associated time delay by means of an impedance boundary called delayed time-domain impedance boundary conditions (D-TDIBCs). The method itself is reviewed in \chap{ch:lit-review}; here we simply observe potential benefits and drawbacks of the ADCBC method against the D-TDIBC method.


% From lit review:
%% "Since the model constants are calculated as a preprocessing step, the envelope of the acoustic response in the frequency domain may essentially be changed arbitrarily, presenting a benefit in case different pass bands are desired. Besides the inevitable drawbacks stemming from: the low-order model's inaccuracy and the requirement of a strongly one-dimensional flow at the boundary to match this model, other drawbacks remain prevalent. For one, no method to visualise acoustics residing in the fictitious, truncated domain is provided, potentially leading to a \emph{black-box} where acoustics are not immediately known. For another, the preprocessing steps are required for each value of $τ$ used. So, if the time delay were to change dynamically during the simulation (e.g. due to an expanding computational domain to ensure the flame remains within), this preprocessing may happen each step, becoming computationally costly."

% - We provide a way to visualise truncated acoustics, which they do not
% - We can add on other boundary treatments simply by including further terms into the specification of L, they instead include different boundary treatments by change the impedance envelope for different frequencies. An acoustic envelope which decays for high frequencies is actually a requirement for D-TDIBC. This has the added benefit that the ADCBC formalism can be added into existing codes already using the NSCBC formalism, under some extra requirements. The ADCBC method propose currently offers no way of modelling upstream or downstream boundary impedance effects. This also means the effectiveness of the method is dependent on the quality of the baseline nonreflecting condition
% - Both methods have very cheap boundary cost, but a pre-processing cost must be paid for D-TDIBC for a given time-delay and envelope. The time delay could change theoretically, although this is not explored in their paper.
% - They have a much lower memory footprint, since no memory of previous steps is required. We have at least two queues for each ADCBC. This could be simply extended to using a queue for each boundary node. As mentioned above, the overall memory footprint of the method should be dominated by interior nodes, especially for three-dimensional simulations
% - In both cases a strongly one-dimensional flow is required at the boundary and only the delay effects under low-amplitude acoustics are modelled





\section{Report Conclusions}





\section{Future Work}

More immediate work ought to be done to further validate the proposed ADCBC method from \chap{ch:delay-bcs}, including investigating the interpolation of $\cl{L}_{1/5}$ values and ensuring that acoustic energy does not trivially increase over long time periods without a flame. Having said that, the method provides a framework for multiple different thermoacoustic applications, particularly for those taking place in long tubes. Thermodiffusive and thermoacoustically unstable flames can already be modelled with the caveat that the rapidly changing flame speeds of thermodiffusive flames likely mean it will exit the DNS domain. To remedy this, dynamic inflow velocities are required which do not couple to the acoustics and counteract the mean flame speed. This is also useful for other, non-thermodiffusive flames which reach secondary instability since the flame speed changes drastically under this regime. Another typical test case and one which is used to test the D-TDIBC method in \cite{douasbin2018DelayedtimeDomainImpedance} is a flame in a strong counterflow, held in place by an adiabatic cylindrical flame holder. Comparisons to their calculated eigenmodes could be found by simulating the same model geometry whilst also providing mode shapes in the fictitious domain. ADCBC should also be easily applicable to flames passing through arrays of cylinders (or other porous geometries). Typically, this is expensive to model due to the task of fitting the discretisation to the complex body and the large scale disparity in thermoacoustics. Utilising the LABFM discretisation in the SUNSET code coupled to low-order ADCBC allows us to model this phenomena for the cost of discretising only the flame and its surrounding hydrodynamic region.

Furthermore, the use of characteristic boundaries allows us to freely impose incoming acoustics through the inflow, $\cl{L}_{5, \rm{imposed}}(t)$ and outflow, $\cl{L}_{1, \rm{imposed}}(t)$ on top of their existing values from the acoustic delay. By implementing an imposed e.g. sinusoidal forcing to model an upstream loudspeaker playing a pure tone, we can trigger secondary instabilities in a model version of \cite{searby1991ParametricAcousticInstability}. This would allow us to more efficiently investigate secondary instability modes and their flame structures. This could be potentially elaborated into three dimensions to try and recreate the wavenumber measurements of \cite{delfin2024DeterminationMethodMarkstein}. Turbulent flow can also be imposed through the VFCBC method established in \cite{guezennec2009AcousticallyNonreflectingReflecting} and an attempt could be made to implement this on top of the existing ADCBC method, as above. Turbulent characteristic outflows would also need to be implemented for compatibility. This may allow us to simulate faster flows in broader domains which are expected to be turbulent, as well as potentially allowing us to observe the self-turbulent flames seen resulting from secondary instability in \cite{searby1992AcousticInstabilityPremixed}.

Thus far, we have gotten away with values of $\abs{R_{\rm{U}/\rm{D}}} = 1$, but more realistic boundaries have impedances which vary with frequency. Could this be explored using some predefined spectral filtering on the delayed $\cl{L}_{1/5}$ values? Practically speaking, a low-pass filter could be used to remove high-frequency noise and improve the quality of observed low frequency modes (at the cost of precluding high-frequency ones). More generally, the non-linear acoustic effects which are not being modelled here could be included by coupling an acoustic solver to the DNS inflow and outflow. Using for example a boundary element method or analytical techniques, acoustics around some up- or downstream geometry can be solved for in tandem with the DNS domain. This could enable us to simulate a flame tube with a porous plug and compare to the analysis of \cite{gaton-perez2025MitigationThermoacousticInstabilities}.



\section{Planning}

\begin{figure}[t]
\centering
\includegraphics[scale=0.5]{assets/graphs/2YR_Gantt.pdf}
\caption{Gantt chart organising tasks for the remaining nine quarters of my PhD.}
\label{fig:gantt}
\end{figure}

There are nine quarters left before my thesis must be written and submitted and the four years of my PhD, beginning January 2024, are over. Currently, I am organising future work into the journal articles (papers) they fit into and timetabling these tasks accordingly:
\setlist[enumerate]{label={\arabic*.}}
\begin{enumerate}
\item \textbf{Paper 1. One quarter needed total.} This paper will focus on the ADCBC method, with examples of an acoustic bump, acoustic standing wave and thermoacoustic flame tube. The method will be explained alongside post-processing for the acoustic domain.
    \begin{enumerate}
    \item Completion of the simulation, post-processing and analysis of a thermoacoustic flame tube for a journal article has yet to be fully done. Primarily due to the analysis, which will be used for further thermoacoustic simulations, half a quarter is needed. Calculations of Rayleigh Indices, RI and acoustic envelopes need to be performed and the aforementioned interpolation issue needs to be investigated
    \item The rest of the quarter will be used to write the paper.
    \end{enumerate}
\item \textbf{Paper 2. Two quarters needed total.} This paper will focus on applications of the ADCBC method to simulations of various physical phenomena, each of which do not require large developments of the ADCBC method.
    \begin{enumerate}
    \item A thermodiffusive and thermoacoustically unstable flame require dynamic inflow non-acoustic velocities. Alongside simulations, post-processing and analysis I estimate this to take half a quarter.
    \item Similarly, a counterflow flame holder flame might require some extra work to implement the correct physical initial conditions for a simulation using ADCBC restarted from an unsteady simulation using non-reflecting NSCBC. I estimate this to take half a quarter too.
    \item The final phenomena is a simulated secondary thermoacoustic instability, which relies on the prior work to change inflow speeds dynamically and impose an additional acoustic field, as mentioned above. I estimate this to take half a quarter too.
    \item The rest of the two quarters will be used to write the paper.
    \end{enumerate}
\item \textbf{Potential third paper. Five quarters needed total.} This paper not yet in focus and its specific contents will be determined by the remaining time and feasibility of each individual task.
    \begin{enumerate}
    \item Simulation of a thermoacoustically unstable flame travelling through arrays of cylinders in a broader domain will likely take a quarter, since I may have to introduce a queue system for each boundary node and perform three-dimensional simulations.
    \item The aforementioned self-turbulent thermoacoustically unstable flame simulation also requires development of the ADCBC to be compliant with turbulent flows and is estimated to take a quarter.
    \item I suspect investigations into more elaborate acoustic modelling will be a deeper rabbit hole, so I estimate it will take about half a year (two quarters).
    \item The rest of the five quarters will be used to write a paper from the resulting research.
    \end{enumerate}
\item \textbf{Thesis. One quarter needed total.}
    \begin{enumerate}
    \item As mentioned, my thesis must be written by the end of 2027 and I estimate it will take roughly one quarter to put together alongside other work.
    \end{enumerate}
\end{enumerate}
This is organised into a Gantt chart in \fig{fig:gantt}, showing how the remaining nine quarters will be used in sequence. Further up-to-date details on this plan are given at \href{https://www.dropbox.com/scl/fi/4w62xppcfkywnr7nm8a6v/plan-Y3.md?rlkey=yx4ez0nmqjaabqseotnzfiqsn&st=lll313fe&dl=0}{\texttt{this link}}.


