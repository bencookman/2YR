\section{Acoustic Bump}

\begin{figure}[t]
\centering
\includegraphics[scale=0.65]{assets/imgs/adcbc_bump_test.pdf}
\caption{The geometry and initialisation for the acoustic bump test case. Not drawn to scale.}
\label{fig:ac-bump-test}
\end{figure}

As our first test case, we model a Gaussian acoustic bump in a two-dimensional tube with two closed ends. The test case is initialised as a small bump in $J_1 \approx p' - ρ c u'$ (where the approximation comes from acoustic linearisation):
\begin{equation}
p_a = A \exp\left( - x^2 / d^2 \right)
\quad \text{and} \quad
u_a = p_a / ρ c
\end{equation}
such that the maximum pressure disturbance is $A = 10$ Pa and diameter $d = 2$ mm. This simple linear approximation also results in a small $J_5$ bump which we ignore. Alternate methods approximating this initial disturbance could also be used, although the provide little relevant improvement. Fluid properties are $u_{\rm{IN}} = 0.2$ m s$^{-1}$, $T_{\rm{IN}} = 298$ K and $p_{\rm{IN}} = 0.1$ MPa comprised of a single species with properties $W = 28$ kg kmol$^{-1}$, $Pr = 0.7$ and $c_p = 1100$ J kg$^{-1}$ K$^{-1}$. The resulting sound speed is $c \approx 348$ m s$^{-1}$. A 6 mm long central DNS region is used as well as up- and downstream acoustic regions modelled using ADCBCs on the inflow and outflow respectively, each modelling 2 mm of truncated tube. These essentially model two one-dimensional acoustic regions to the left and right of the flame. A diagram for this test case is shown in \fig{fig:ac-bump-test}. A sample rate of $δt_{\rm{sample}} \approx 0.25$ ms is used for both ADCBCs and we use \nth{0} order interpolation (constant sample values) at the moment, for reasons discussed later. The horizontal top and bottom boundaries used in the DNS region are symmetric mirror boundaries, 1 mm apart. With the linear acoustics equations in the acoustic regions being non-dispersive, we expect the Gaussian to retain its shape after each up- and downstream bounce. The full reacting Navier-Stokes equations are solved in the DNS region as a precursor to the a flame being present in the DNS region. Only in the DNS region should viscous effects alter the wave packet's shape and even in this case, only slightly after many bounces back and forth the 1 cm long domain. The ADCBCs work almost identically when the viscous inert, isothermal equations, which are also implemented into the SUNSET code, are solved. Note that the response to transverse effects are not being tested here because the acoustic is entirely one-dimensional. As mentioned above, we should in fact be modelling the convected wave equation, not the wave equation for quiescent fluids. But, since our Mach number, $\Ma < 10^{-3}$ is so low in this case the convective effects are deemed to be asymptotically unimportant.

\begin{figure}[t]
\centering
\includegraphics[scale=0.36]{assets/graphs/AC_BUMP_first_bounces_comp.png}
\caption{Acoustic pressure fields in the first 55 ms in the DNS region. Only have the top half of each DNS region is shown for comparison.}
\label{fig:ac-bump-dns}
\end{figure}

Pressure fields for the first bounces off the up- and downstream acoustic boundary are shown in \fig{fig:ac-bump-dns}. It seems after a single bounce off each end that the ADCBCs are adequately sampling and reintroducing the acoustic bump with negligible changes to its shape. The sample rate means that roughly $d / δt_{\rm{sample}} = 8$ samples are used to sample the first standard deviation of width of the Gaussian. This reconstruction of acoustic shape is impressive given the lack of interpolation (\nth{0} order interpolation means using constant sample values) used. After 750 ms, the acoustic should have been sampled by the ADCBCs and reentered the DNS domain many times. The pressure and velocity fields are shown in the DNS region at this time in \fig{fig:ac-bump-dns-late}.

\begin{figure}[t]
\centering
\includegraphics[scale=0.36]{assets/graphs/AC_BUMP_ndt=150e-4_comp.png}
\caption{$p_a$ and $u$ fields after 750 ms.}
\label{fig:ac-bump-dns-late}
\end{figure}

% Acoustic bump results
% - Show result of a Gaussian bump leaving the domain and reentering after 1 bounce, 2 bounce (each boundary) and 100 bounces for different sample periods and interpolation orders?
% - For thin regions like this the averaging along the boundary makes sense, but as the DNS domain widens, it seems reasonable that this averaging would result in more and more inaccuracy (and difficulty due to instability...), so the similar method where queues are used for each boundary node should instead be used in wider domains. For all the cases in this report, the averaging procedure is used and we find good results regardless.



\begin{figure}[t]
\centering
\includegraphics[scale=0.65]{assets/imgs/wave-sampling-comparison.pdf}
\caption{SAMPLING}
\label{fig:wave-sampling}
\end{figure}

% Note that the pressure drift would be fixed by a conservative method, but the destruction of the wave shape would not be

% Now is a good time to bring up the sampling error
% - \fig{fig:wave-sampling} shows a representative analytic L_{1/5} field for two acoustic bumps (which is roughly the derivative of the Gaussian). Despite the same linear interpolation and sample period being used, the sampled $L$ values on the left are much worse than on the right. The same could be true for any number of samples at a constant interpolation order and sample period, where the change in phase of the sampling with the wave will always cause some inaccuracy in the integrated acoustic field. This inaccuracy results in a change to the shape of the Gaussian, e.g. a skew to one side, which feeds back into this issue such that the waves eventually fully break down
% - For the acoustic bump in this small domain this seems to happen relatively quickly because the associated acoustic frequencies of this system are higher. But, if the bump remains the same size and the tube is made longer, less bounces happen and the sampling error grows slower
% - Furthermore, there is the added effect that natural wavenumbers of these longer tubes will be lower, meaning that individual waves can be sampled better. This leads to an asymptotic behaviour of O(l^2) for values of error productions which are small (DO SOME MATHS?).




% Different interpolation orders:
% - Constant interpolation maintains the quality of acoustic bump surprisingly well, whilst the non-constant interpolation seems to introduce some small error on every reentry of the acoustic bump
% - assuming it isn't the fault of a mistake in the sunset implementation, it seems like it shouldn't be caused by the boundary averaging procedure given that the fields are initialised and remain virtually one-dimensional the whole time. It is likely the rest of the method then regardless of averaging which contains some error which cancel out when constant values are used but doesn't for non-constant values?






\section{Acoustic Standing Wave}

\subsection{Test Case}

% Describe test case
% - An acoustic standing wave. DNS region calculated by initialising the pressure field to a sinusoid, and the upstream/downstream regions are initialised by specifying values of L within the queue initially. Usually these queues would have been initialised as empty -- instead they are given corresponding L_1/5 values
% - other than this the code is ran in the same way



\subsection{Results}

\subsubsection{Overcoming Inflow Instability}

% Instability issue!
% - We observed that under certain boundary discretisations, a wringing instability of u develops at the inflow after many acoustic periods. This grows exponentially, suggesting the instability behaves linearly, until the vorticity produced within the DNS region destroys the solution
% - Show image!!
% - This instability can be reduced by increasing the coefficient of the hyperviscosity filter at the inflow boundary nodes. Specifically, by increasing the hyperviscosity tangential to the inflow, the wringing mode is dampened to the point that it cannot grow
% - for us, we..
% - This instability may or may not show up when multiple queues are used for each inflow/outflow, so this needs to be investigated. It appears to be largely a result of the boundary averaging procedure.

% We use SI units (1 / m)
\begin{figure}[t]
\centering
\includegraphics[scale=0.36]{assets/graphs/u-inflow-instab.png}
\caption{INFLOW INSTAB}
\label{fig:inflow-instab}
\end{figure}


\subsubsection{Sampling Error}

% Sampling error
% - We can also see the sampling instability in the standing wave results as a q-wave \cite{poinsot2001TheoreticalNumericalCombustion} which is largest at maximum values of L???
% - Show image/s. Easiest to see when instability is not present
% - We can find similar results for standing waves as a bump in terms of interpolation order?
% - Once again, for tubes which are long enough this should not be a concern due to the quadratic growth of period with length. This is consistent with our results in the next section

% We use SI units (1 / m)
\begin{figure}[t]
\centering
    \begin{subfigure}{0.49\textwidth}
    \centering
    \includegraphics[scale=0.325]{assets/graphs/double-instab.0003.png}
    \caption{}
    \label{fig:qwave1}
    \end{subfigure}
    \begin{subfigure}{0.49\textwidth}
    \centering
    \includegraphics[scale=0.325]{assets/graphs/double-instab.0004.png}
    \caption{}
    \label{fig:qwave2}
    \end{subfigure}

\vspace*{0.5em}

    \begin{subfigure}{0.49\textwidth}
    \centering
    \includegraphics[scale=0.325]{assets/graphs/double-instab.0005.png}
    \caption{}
    \label{fig:qwave3}
    \end{subfigure}
    \begin{subfigure}{0.49\textwidth}
    \centering
    \includegraphics[scale=0.325]{assets/graphs/double-instab.0006.png}
    \caption{}
    \label{fig:qwave4}
    \end{subfigure}
\caption{Q WAVE}
\label{fig:qwave}
\end{figure}



\subsubsection{Moving DNS Domain}

% Moving domain results
% - shows validation that the delay time can indeed change
% - time scale for delay change is much longer due to the low Mach number
% - Show two images at separate times where the domain has moved but the acoustic remains well-resolved




\section{Thermoacoustically Unstable Flame}

\subsection{Test Case}

\begin{figure}[t]
\centering
    \begin{subfigure}{0.99\textwidth}
    \centering
    \includegraphics[scale=0.25]{assets/graphs/flame-sim-discretisation.png}
    \caption{}
    \label{fig:disc1}
    \end{subfigure}

\vspace*{0.5em}

    \begin{subfigure}{0.99\textwidth}
    \centering
    \includegraphics[scale=0.25]{assets/graphs/flame-sim-discretisation_zoom.png}
    \caption{}
    \label{fig:disc2}
    \end{subfigure}
\caption{DNS COMPUTATIONAL DOMAIN}
\label{fig:disc}
\end{figure}

% We now study a thermoacoustically unstable flame in a reflected DNS domain shown in fig ..
% Flame properties are: single step irreversible reaction with fluid properties constant in reactants and products, q = 6, Ze = 5, S_L = u_in = 0.2 m / s, Le = 1, with derived laminar flame thickness l_L = 0.216 mm
% Even though LABFM enables variable resolutions, we use constant discretisation length scale for these simulations as we don't know where the flame will end up a priori. A discretisation length scale of $s = 18$ μm is used and w = 2mm.
% Show the discretisation used zoomed in and out!
% It may seem like a large jump to go from inert, essentially isothermal acoustics to fully reacting flame-acoustic interactions, but the sunset code is designed with flame simulations in mind, so the simulations are 'easy' to perform having already implemented ADCBC.
% (Maybe there are better intermediate test cases? D-TDIBC don't seem to need them)



\subsection{Acoustic Eigenmodes}

% Prediction from Eigenmodes
% - First we predict what results we get in a one-dimensional closed-open tube of the same length under the linear acoustic approximation, modelling the flame as a discontinuity which does not interact with the acoustics.
% First take the non-dimensionalised equations for compressible, inviscid fully non-linear flow:
% ... (continue on with maths)

\begin{figure}[t]
\centering
\includegraphics[scale=0.35]{assets/graphs/r=7_harmonics_both.pdf}
\caption{$r = 7$, LEFT: , RIGHT: divided by }
\label{fig:flame-harmonics}
\end{figure}

\begin{figure}[t]
\centering
\includegraphics[scale=0.35]{assets/graphs/r=7_xf=05_complex_harmonics.pdf}
\caption{$r = 7, x_f = 0.5$, CAPTION}
\label{fig:flame-harmonics-complex}
\end{figure}

% For a flame which is .. and .., we expect harmonics of frequencies .. provided that the acoustics remain linear and are not interacting with the flame
% Note that this only accounts for the modes of a stationary density jump. If the flame is moving instead, the acoustic modes in the tube change with the moving flame. Hence the acoustics in the tube will be much more complex in this case in a way which is not predicted by this model. More complex control diagrams may be required to model the moving system instead.




\subsection{Results}

% Show some fields at an example time for both simulations

% stats data fft postprocessing/windowing

% spectrogram results

% frequencies

% mode structures!



\begin{figure}[t]
\centering
\includegraphics[scale=0.35]{assets/graphs/fft-windowing.png}
\caption{[PLACEHOLDER IMG] FFT WINDOWING}
\label{fig:windowing}
\end{figure}

\begin{figure}[t]
\centering
\includegraphics[scale=0.35]{assets/graphs/spectrogram.png}
\caption{[PLACEHOLDER IMG] SPECTROGRAM}
\label{fig:spectrogram}
\end{figure}

\begin{figure}[t]
\centering
\includegraphics[scale=0.35]{assets/graphs/pp-tones.png}
\caption{[PLACEHOLDER IMG] POST-PROCESSED 1/4 AND 3/4 WAVE EVIDENCE}
\label{fig:pp-tones}
\end{figure}


