\begin{abstract}

In this report, we tackle the problem of expensive thermoacoustic simulations by introducing a new boundary conditions (BCs) scheme based off the Navier-Stokes characteristic BCs (NSCBCs) formulation. This truncates a flame tube by restricting DNS to the flame and its surrounding hydrodynamics and allows tube acoustics to be modelled using a simple delay on acoustics leaving this DNS domain. Code schematics for this method, dubbed the acoustic delay characteristic BCs (ADCBCs), are presented as they relate to implementation into the NSCBC scheme. Instabilities resulting from the acoustic-DNS coupling can be reduced by increased tangential filtering at the boundaries, although this is unlikely to be an issue when one-dimensional approximations are made at each boundary node. Simulations are then performed using these boundary conditions and show that solutions show excellent agreement with the acoustic eigenmodes of a one-dimensional flame tube system with no damping elements. Comparisons are drawn between ADCBCs and the existing delayed time-domain impedance BCs (D-TDIBC) which instead models the expected acoustic impedance at the truncated boundary according to the acoustic delay observed. Various benefits of the method include: the massive increase to computational cost as only one part of the tube is discretised; the delay times can trivially change allowing the flame region to be translated along the tube's length; ADCBCs are implemented as an additional layer on top of NSCBCs with few additional requirements being made, allowing parts of the NSCBCs to be modularly reintroduced; only a simple one-dimensional linear model for acoustics is currently being used, although other nonlinear or higher-dimensional models could be used; modelling could be done to model non-trivial upstream and downstream impedances.

\end{abstract}
