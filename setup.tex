%%%%% This is a generic preamble for article-like LaTeX documents. I can copy and paste this file into new documents to quickly make LaTeX styling which I like and is consistent with other documents I've typeset

\usepackage[a-3u]{pdfx}                 % PDF document properties
\usepackage{graphicx,psfrag}            % For postscript graphics files
    % \graphicspath{ {./images/} }
\usepackage{amsmath}                    % assumes amsmath package installed
    \allowdisplaybreaks[1]              % allow eqnarrays to break across pages
\usepackage{amssymb}                    % assumes amsmath package installed
\usepackage{url}                        % format hyperlinks correctly
\usepackage{rotating}                   % allow portrait figures and tables
\usepackage{multirow}                   % allows merging of rows in tables
\usepackage{lscape}                     % allows pages to be typeset in landscape mode
\usepackage{tabularx}                   % allows fixed width tables
\usepackage{verbatim}                   % enhanced version of built-in verbatim environment
\usepackage{footnote}                   % allows more control over footnote environments
\usepackage{float}                      % allows H option on floats to force here placement
\usepackage{booktabs}                   % improve table line spacing
\usepackage{lipsum}                     % for adding dummy text here
\usepackage[base]{babel}                % required for lisum package
\usepackage{subcaption}                 % for multiple sub-figures in a single float
\usepackage{siunitx}                    % add SI units
% \usepackage[dvipsnames]{xcolor}         % More colours
\usepackage{physics}
\usepackage{tabto}                      % \tab command
\usepackage{ragged2e}                   % Better ragged left
\usepackage{array}
\usepackage{caption}
\usepackage{mathtools}
\usepackage{bbm}
\usepackage{enumitem}
\usepackage{empheq}
\usepackage{gensymb}
\usepackage{textcomp}
\usepackage[bottom]{footmisc}           % Places footnotes at the bottom of the page
\usepackage{esvect}                     % For \vv
% \usepackage[a4paper, margin=1cm, tmargin = 1.5cm, bmargin=2cm, footskip = 1cm]{geometry}
\usepackage[skip = 10pt]{parskip}
\usepackage{indentfirst}                % I want my first paragraphs to always be indented
\usepackage{stackengine}
\usepackage{scalerel}
\usepackage{stmaryrd}
\usepackage{plimsoll}
% \usepackage{titlesec}
% \usepackage{MnSymbol}
\usepackage{accents}
\usepackage[super]{nth}



% \geometry{margin=1cm}
% \titlespacing*{\chapter}{0pt}{-40pt}{10pt}

\newenvironment{boxequ}{\empheq[box=\widefbox]{equation}}{\endempheq}
\newenvironment{boxali}{\empheq[box=\widefbox]{align}   }{\endempheq}
\newenvironment{boxaliat}[1]{\empheq[box=\widefbox]{alignat=#1}}{\endempheq}

\newcommand*{\ndt}[1]{%
  \accentset{\mbox{\large\bfseries .}}{#1}}
\newcommand*{\nddt}[1]{%
  \accentset{\mbox{\large\bfseries .\hspace{-0.25ex}.}}{#1}}

% \usepackage{scalerel,stackengine}
% \stackMath
% \newcommand\widecheck[1]{%
% \savestack{\tmpbox}{\stretchto{%
%   \scaleto{%
%     \scalerel*[\widthof{\ensuremath{#1}}]{\kern-.6pt\bigwedge\kern-.6pt}%
%     {\rule[-\textheight/2]{1ex}{\textheight}}   %WIDTH-LIMITED BIG WEDGE
%   }{\textheight}% 
% }{0.5ex}}%
% \stackon[1pt]{\displaystyle #1}{\scalebox{-1}{\tmpbox}}%
% }
% \renewcommand\widehat[1]{%
% \savestack{\tmpbox}{\stretchto{%
%   \scaleto{%
%     \scalerel*[\widthof{\ensuremath{#1}}]{\kern-.6pt\bigvee\kern-.6pt}%
%     {\rule[-\textheight/2]{1ex}{\textheight}}   %WIDTH-LIMITED BIG HAT
%   }{\textheight}% 
% }{0.5ex}}%
% \stackon[1pt]{\displaystyle #1}{\scalebox{-1}{\tmpbox}}%
% }

\newcommand{\sus}[1]{$^{\mbox{\scriptsize #1}}$}
\newcommand{\sub}[1]{$_{\mbox{\scriptsize #1}}$}
\newcommand{\chap}[1]{Chapter~\ref{#1}}
\newcommand{\sect}[1]{Section~\ref{#1}}
\newcommand{\fig}[1]{Fig.~\ref{#1}}
\newcommand{\tabl}[1]{Table~\ref{#1}}
\newcommand{\equ}[1]{(\ref{#1})}
\newcommand{\appx}[1]{Appendix~\ref{#1}}

\newcommand{\blue}[1]{\colorlet{saved-blue}{.}\color{NavyBlue}#1\color{saved-blue}}
\newcommand{\red}[1]{\colorlet{saved-red}{.}\color{Red}#1\color{saved-red}}
\newcommand{\green}[1]{\colorlet{saved-green}{.}\color{PineGreen}#1\color{saved-green}}
\newcommand{\purple}[1]{\colorlet{saved-purple}{.}\color{Plum}#1\color{saved-purple}}
\newcommand{\orange}[1]{\colorlet{saved-orange}{.}\color{YellowOrange}#1\color{saved-orange}}
\newcommand{\brown}[1]{\colorlet{saved-brown}{.}\color{Brown}#1\color{saved-brown}}
\newcommand{\pink}[1]{\colorlet{saved-pink}{.}\color{CarnationPink}#1\color{saved-pink}}


\renewcommand{\a}{\alpha}
\renewcommand{\b}{\beta}
\newcommand{\g}{\gamma}
\renewcommand{\d}{\delta}
\newcommand{\e}{\epsilon}
\newcommand{\z}{\zeta}
\renewcommand{\th}{\theta}
\renewcommand{\i}{\iota}
\renewcommand{\k}{\kappa}
\renewcommand{\l}{\lambda}
\renewcommand{\r}{\rho}
\renewcommand{\t}{\tau}
\newcommand{\s}{\sigma}
\renewcommand{\u}{\upsilon}
\renewcommand{\r}{\rho}
\newcommand{\vr}{\varrho}
\newcommand{\w}{\omega}
\newcommand{\f}{\varphi}
\newcommand{\G}{\Gamma}
\newcommand{\D}{\Delta}
\newcommand{\Th}{\Theta}

\newcommand{\fa}{\forall\:}
\newcommand{\fe}{\exists\:}

\renewcommand{\rm}{\mathrm}
\newcommand{\bb}[1]{\mathbb{#1}}
\newcommand{\cl}[1]{\mathcal{#1}}
\newcommand{\fk}[1]{\mathfrak{#1}}

\newcommand{\Le}{\rm{Le}}
\renewcommand{\Pr}{\rm{Pr}}
\newcommand{\Ma}{\rm{Ma}}
\newcommand{\Ze}{\rm{Ze}}
\newcommand{\Mk}{\cl{M}}
\newcommand{\Fr}{\rm{Fr}}
\newcommand{\Pe}{\rm{Pe}}
\newcommand{\rhs}{\rm{RHS}}

\renewcommand{\vec}[1]{\mathbf{#1}}
\newcommand{\lap}{\D}

\def\stacktype{S}
\newcommand{\hvec}[1]{\stackon[-0.5pt]{#1}{\scaleobj{0.9}{\rightharpoonup}}}
% \newcommand{\nhat}[1]{\stackon[-3.5pt]{#1}{\scaleobj{0.9}{\textasciicircum}}}
\newcommand{\nhat}[1]{\widehat{#1}}
\newcommand{\uvec}[1]{\nhat{\vec{#1}}}
\newcommand{\ftvar}[1]{\stackon[0.5pt]{#1}{\scaleobj{0.75}{\sim}}}
\newcommand{\vvt}[1]{\vv{\vv{#1}}}
\newcommand{\und}[1]{\underline{#1}}
\newcommand{\undt}[1]{\und{\und{#1}}}

\newcommand{\ft}{\cl{F}}
\newcommand{\ift}{\cl{F}^{-1}}
\DeclareMathOperator{\DFT}{DFT}
\DeclareMathOperator{\IDFT}{IDFT}

\newcommand{\angs}[1]{\left\langle #1 \right\rangle}
\newcommand{\ceil}[1]{\left\lceil #1 \right\rceil}
\newcommand{\floor}[1]{\left\lfloor #1 \right\rfloor}
\newcommand{\bbra}[1]{\left\llbracket \mspace{2mu} #1 \mspace{2mu} \right\rrbracket}
\newcommand{\aang}[1]{\left\llangle #1 \right\rrangle}
% \newcommand{\ccor}[1]{\ullcorner #1 \ulrcorner}
\newcommand{\ccor}[1]{\boldsymbol{\ullcorner} #1 \boldsymbol{\ulrcorner}}

\renewcommand{\dim}{N_{\rm{D}}}

\newcommand{\mDv}[1]{\frac{\mathrm{D}}{\mathrm{D} #1}}
\newcommand{\mdv}[2]{\frac{\mathrm{D} #1}{\mathrm{D} #2}}
\newcommand{\gbar}{\bar\g}

\newcommand{\tang}{\parallel}



\captionsetup{width=\textwidth} 

\newcommand*\widefbox[1]{\fbox{\hspace{1.5mm}#1\hspace{1.5mm}}}
\setlength\fboxsep{3mm}


% Note backref=true adds a page number (and hyperlink) to each reference so you can easily go back from the references to the main document. You may prefer backref=false if you need to stick strictly to a given reference style
\usepackage[style = ieee, backend = biber, backref = false, hyperref = auto]{biblatex}
\renewcommand*{\bibfont}{\small}
\DefineBibliographyStrings{english}{backrefpage = {cited on p\adddot},  backrefpages = {cited on pp\adddot}}