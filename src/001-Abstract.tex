\begin{abstract}

In this report, we tackle the problem of expensive thermoacoustic simulations by introducing a new boundary scheme based off the Navier-Stokes Characteristic Boundary Conditions (NSCBCs) formulation. This truncates a flame tube by restricting Direct Numerical Simulation (DNS) to the flame and its surrounding hydrodynamics and allows tube acoustics to be modelled using a simple delay on acoustics leaving this DNS domain. Code schematics for this method, dubbed the Acoustic Delay Characteristic Boundary Conditions (ADCBCs), are presented. A method is provide to reconstruct the acoustic field in the fictitious region. Simple fluid simulations reproduce accurate reflection of a single acoustic bump and standing wave. A flame simulation is then performed which shows excellent agreement to the acoustic eigenmodes of a one-dimensional flame tube system with no damping elements. Various benefits of the method include: the massive increase to computational cost as only one part of the tube is discretised; the delay times can trivially change allowing the flame region to be translated along the tube's length; ADCBCs are implemented as an additional layer on top of NSCBCs with few additional requirements being made, allowing parts of the NSCBCs to be modularly reintroduced; only a simple one-dimensional linear model for acoustics is currently being used, although other nonlinear or higher-dimensional models could be used; modelling could be done to model non-trivial upstream and downstream impedances.

\end{abstract}
